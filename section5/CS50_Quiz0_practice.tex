\documentclass[12pt]{exam}
\usepackage[utf8]{inputenc}

\usepackage[margin=1in]{geometry}
\usepackage{amsmath,amssymb}
\usepackage{multicol}
\usepackage{listings}
\usepackage{courier}

\lstset{basicstyle=\footnotesize\ttfamily, frame=single, showstringspaces=false, numbers=left}

\newcommand{\class}{Robbie Gibson-Practice Questions}
\newcommand{\examnum}{Quiz 0}

\pagestyle{head}
\firstpageheader{}{}{}
\runningheader{\class}{\examnum\ - Page \thepage\ of \numpages}
\runningheadrule

\begin{document}

\noindent
\begin{tabular*}{\textwidth}{l @{\extracolsep{\fill}} r @{\extracolsep{6pt}} l}
\textbf{\class} & \textbf{Name:} & \makebox[2in]{\hrulefill}\\
\textbf{\examnum} &&\\
\end{tabular*}\\
\rule[2ex]{\textwidth}{2pt}

\begin{questions}

\question
Angela loves Catalan numbers and their cool properties in combinatorics. Write a function to find the nth Catalan number. The Catalan numbers are expressed as:

\begin{align*}
\frac{(2n)!}{(n+1)!n!}
\end{align*}

{\it Hint: Write a second function to find the factorial of a given number.
Try making it recursive!}

\makeemptybox{4in}

\newpage

\question Below is a C program that sets a lot of variables.
In the table, write the value of each variables after the given line has been executed.
You should use {\tt \&x} and the like when writing the value of the pointers.

\begin{lstlisting}[language=c, mathescape]
int main(int argc, char** argv)
{
    int x = 2, y = 3, z = 4;
    int$^*$ a = &x;
    int$^*$ b = &y;
    int$^*$ c = &z;
    
    y += $^*$a;
    z = y + $^*$b;
    $^*$a = x + $^*$b;
    c = b;
    $^*$b = $^*$c;
    y = z + $^*$c;
    
    return 0;
}
\end{lstlisting}

{\tt
\begin{tabular}{r|c|c|c|c|c|c}
Line & x & y & z & a & b & c \\
6    &   &   &   &   &   &   \\
8    &   &   &   &   &   &   \\
9    &   &   &   &   &   &   \\
10   &   &   &   &   &   &   \\
11   &   &   &   &   &   &   \\
12   &   &   &   &   &   &   \\
13   &   &   &   &   &   &   \\
\end{tabular}
}

\question
True or False:
\begin{enumerate}
\item Malloc allocates memory on the stack.
\item A string in C is an array.
\item It's possible to sort an array in $O(n \log{n})$ time.
\item Angela creates and declares an int array, but doesn't initialize it. It's filled with 0's.
\item Typing {\tt next} in gdb will step into a function called on that line.
\item The size of an {\tt int} on the CS50 Appliance is 4 bytes.
\item 4 bytes is 36 bits.
\end{enumerate}

\newpage

\question
Angela writes some code to determine if a number is prime. Unfortunately, it's buggy and that makes her sad.

How can you fix this implementation?

\begin{lstlisting}[language=c, mathescape]
#include <stdio.h>
#include <cs50.h>

int main(int argc, char** argv) 
{
    printf("Enter value of N : ");
    int n = GetInt();
  
    int flag = 0;
    int i;
    for (i = 2; i <= (n / 2); ) 
    { 
        if (n % i == 0) /* If true n is divisible by i */
        {
            flag = 0;
            break;
        }
    }
 
    if (flag)
    {
        printf("%d is prime\n", n);
    }
    else
    {    
        printf("%d has %d as a factor\n", n, i);
    }
    return 0;
}
\end{lstlisting}

\question
How many different, unique values can you represent with 4 bits?

\question 
Angela's arguing with Robbie about how to best sort a large {\tt int} array. Angela thinks Mergesort is faster, but Robbie keeps saying Bubblesort is faster. Who's correct, what are the best-case and worst-case runtimes, and why are the two runtimes different?

\makeemptybox{2in}

\newpage

\question
Angela says a pointer is a variable whose value is the direct address of a memory location. 
\begin{enumerate}
\item Is there anything wrong with this definition? What's wrong with it?
\item Declare a pointer.
\item Why are pointers useful?
\end{enumerate}
\makeemptybox{2in}


\question
Write your own string compare function.
The real {\tt strcmp()} has complicated return values, but yours should {\tt return 1} if the strings are equal and {\tt return 0} otherwise.
{\bf Use pointer arithmetic to look at each {\tt char} in the string and NOT array indexing.}
Also, assume that you can't use any other functions (like {\tt strlen()}).

{\tt int strcmp(char$^*$ str1, char$^*$ str2)

\{ }

\end{questions}

\end{document}